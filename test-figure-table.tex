\chapter{图表 vs 浮动体}

\section{title}
Myriad,英语单词,意为「无数的」。同时,「Myriad」也是一款字体的名字。
由罗伯特·斯林巴赫(Robert Slimbach,1956年-)和卡罗·图温布利
(Carol Twombly,1959年-)在1990年到1992年期间以 Frutiger 字体为蓝本
为 Adobe 公司设计。 Myriad 是早期数码字体时代的先驱,伴随着技术的成长
一路走来。

\begin{figure}[h]
  \centering
  \includegraphics[width=3cm]{../testfiles/support/fudan-emblem.pdf}
  \includegraphics[width=4cm]{../testfiles/support/fudan-emblem-new.pdf}
  \caption{Multiple Master 是 Type 1字体格式的扩展部分。Type 1 是利用
    PostScript 语言描述字形信息的字体系统。Type 1字体是第一款矢量字体
    (outline font),通过二维坐标系中的关键点和三次贝塞尔曲线描述字体
    的边缘,在屏幕显示和输出时,在光栅图像处理器内,根据字号大小计算
    出字体边缘(栅格化)。}
\end{figure}

如今,它更多地和我们相见在显示屏幕上。当然,还有那著名的标榜设计的
电子品牌。1992 年,耗时两年开发的 Myriad 终于发布了历史上第一个版本:
Myriad MM。

\section{title}
这款温和且具有良好可读性的人文主义无衬线字体,集诸多当时最新的数字
字体技术于一身。 后缀 MM,意为 Multiple Master,没有找到对应的中文
译名,我们权且称之为「多母板技术」。Myriad 是最早采用 Multiple Master
技术的无衬线字体之一。这项技术的原理是在坐标轴(Axis)的区间两端设计
极限母板,中间的变量则采取线性或非线性变化,对于字体来说,字型的宽度、
粗细甚至有无衬线,都可以在坐标轴上设置。此外,MM 技术还提供了在小字号
下屏幕显示的视觉修正(Optical Adjustment),也就是说,同一款字体,在
小字号时,其字间距和笔画粗细,会被适当地放大。而衬线字体,随着字号的
变小,衬线会相对变粗。视觉修正可以提高小字号字体的识别性,对于远低于
印刷分辨率的电脑屏幕来说,也具有重要意义。

\begin{table}[h]
  \centering
  \caption{一个 normal 表格}
  \begin{tabular}{ccc}
    \hline
    \bfseries 功能 & \bfseries 环境 & \bfseries code \\
    \hline
    表格 & tabular & \ttfamily \backslash begin\{tabular\} ... \backslash end\{tabular\} \\
    插图 & figure  & \ttfamily \backslash begin\{figure\}  ... \backslash end\{figure\}  \\
    居中 & center  & \ttfamily \backslash begin\{center\}  ... \backslash end\{center\}  \\
    \hline
  \end{tabular}
\end{table}

在 Multiple Master 的时代,字号是从6pt到72pt之间非线性设置的。这一传统
保留到了今天 Truetype 和 Opentype 的 Single Master 时代。Adobe 软件的
字体下拉菜单,仍然只显示6到72pt 的字号。
